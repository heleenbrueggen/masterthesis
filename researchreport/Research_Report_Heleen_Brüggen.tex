\documentclass[10pt, a4paper, titlepage]{article}
\usepackage[margin=1in]{geometry}
\usepackage{graphicx, latexsym}
\usepackage{titling}
\setlength{\droptitle}{-25em}
\renewcommand{\maketitlehooka}{\Large}
\usepackage{setspace}
\usepackage{amssymb, amsmath, amsthm}
\usepackage[export]{adjustbox}
\usepackage{bm}
\usepackage{wrapfig}
\usepackage{epstopdf}
\usepackage{microtype}
\usepackage[hidelinks]{hyperref}
\usepackage{titling}
\hypersetup{
    pdftitle={Research Report Heleen Brüggen},
    pdfauthor={Heleen Brüggen},
    pdfsubject={Research Report Heleen Brüggen},
    pdfkeywords={},
    bookmarksnumbered=true,
    bookmarksopen=true,
    bookmarksopenlevel=1,
    colorlinks=false,
    pdfstartview=Fit,
    pdfpagemode=UseNone
}

\singlespacing

\begin{document}
\begin{titlingpage}
\begin{center}
\Huge\textbf{Master Research Report:  \\ Multilevel Multivariate Imputation by Chained Equations through Bayesian Additive Regression Trees} \\
\Large\textit{Methodology and Statistics for the Behavioural, Biomedical and Social Sciences}

\vspace{.5cm}

\normalsize\textit{Heleen Brüggen}

\vspace{15cm}

\begin{minipage}{0.5\textwidth}
\begin{flushleft}

\textbf{Word count:} \\
\textbf{Candidate Journal:} \\
\textbf{FETC Case Number:} \\
\textbf{Supervisors:} \\
MSc. T. Volker \\
Dr. G. Vink \\
 MSc. H. Oberman
\end{flushleft}
\end{minipage}%
\begin{minipage}{0.5\textwidth}
\begin{flushright}

... \\
Computational Statistics \& Data Analysis \\
23-1778 \\
------------------------\\
Utrecht University \\
Utrecht University \\
Utrecht University
\end{flushright}
\end{minipage}

\end{center}
\end{titlingpage}

\newpage

\section{Introduction}

\subsection{Introducing missing data, multiple imputation \& multilevel data structure}
Incomplete data is a common challange in many fields of research. A common approach for dealing with incomplete data is to remove all missing values from the data. However, this could possibly lead to biased results if the data is not Missing Completely At Random (MCAR) \cite{buurenFlexibleImputationMissing2018, kang2013, enders2017, austin2021}. MCAR is one of the missing data mechanisms described by Rubin \cite{rubin1976}. Where MCAR means the cause of the missing data are unrelated to the data, Missing At Random (MAR) that it is related to observed data and Missing Not At Random (MNAR) that it is related to unobserved data \cite{buurenFlexibleImputationMissing2018, rubin1976}. Furthermore, other approaches to dealing with incomplete data include: pairwise deletion, mean imputation and regression imputation, which also yield biased results \cite{buurenFlexibleImputationMissing2018}.

Multiple imputation (MI) is considered a valid method for dealing with incomplete data \cite{mistlerComparisonJointModel2017, buurenFlexibleImputationMissing2018, enders2017, burgette2010, austin2021, audigier2018, vanbuuren2007, grund2021}. MI imputes each missing value more than once, thereby considering necessary variation associated with the missingness problem. The multiply imputed data sets are analyzed, and the corresponding inferences are pooled according to Rubin's rules \cite{buurenFlexibleImputationMissing2018, austin2021, rubin1987}.
Generally, multiple imputation operates under two frameworks: joint modeling and fully conditional specification \cite{mistlerComparisonJointModel2017, buurenFlexibleImputationMissing2018}. Joint modeling (JM) employs a multivariate data distribution and regression model to impute missing values \cite{buurenFlexibleImputationMissing2018, enders2018}. Fully conditional specification (FCS), or chained equations, iteratively imputes one variable with missing values at a time through conditional univariate distributions \cite{enders2018, buurenFlexibleImputationMissing2018}. The JM and FCS approaches are extended to a multilevel imputation context, where data is structured in a hierarchical way (students nested within classes) \cite{mistlerComparisonJointModel2017}.

\subsection{Literature review (difficulty of imputing multilevel data)}

\subsection{Revelence of research}

\subsection{Research question}

\subsection{Hypotheses}

\section{Method}


\section{Results}

\newpage
\bibliography{thesis}
\bibliographystyle{apalike}

\end{document}

