\documentclass[10pt, a4paper, titlepage]{article}
\usepackage[margin=1in]{geometry}
\usepackage{graphicx, latexsym}
\usepackage{titling}
\setlength{\droptitle}{-25em}
\renewcommand{\maketitlehooka}{\Large}
\usepackage{setspace}
\usepackage{amssymb, amsmath, amsthm}
\usepackage[export]{adjustbox}
\usepackage{bm}
\usepackage{wrapfig}
\usepackage{epstopdf}
\usepackage{microtype}
\usepackage[hidelinks]{hyperref}
\usepackage{titling}
\usepackage{multirow}
\usepackage[labelfont=bf]{caption}
\captionsetup{format=hang,justification=raggedright,singlelinecheck=false}
\usepackage[table,xcdraw]{xcolor}
\usepackage{colortbl}
\usepackage{lscape}
\usepackage{float}
\usepackage[sort&compress,round,semicolon,authoryear]{natbib}
\usepackage{bookmark}
\usepackage{listings}
\usepackage{xcolor}
\definecolor{codegreen}{rgb}{0,0.6,0}
\definecolor{codegray}{rgb}{0.5,0.5,0.5}
\definecolor{codepurple}{rgb}{0.58,0,0.82}
\definecolor{backcolour}{rgb}{1,1,1}
\lstdefinestyle{mystyle}{
    backgroundcolor=\color{backcolour},   
    commentstyle=\color{codegreen},
    keywordstyle=\color{magenta},
    numberstyle=\tiny\color{codegray},
    stringstyle=\color{codepurple},
    basicstyle=\ttfamily\footnotesize,
    breakatwhitespace=false,         
    breaklines=true,    captionpos=t,                    
    keepspaces=true, numbers=left,                    
    numbersep=5pt, showspaces=false,                
    showstringspaces=false, showtabs=false, tabsize=2}
\lstset{style=mystyle}
\hypersetup{
    pdftitle={Master Thesis Heleen Brüggen},
    pdfauthor={Heleen Brüggen},
    pdfsubject={Master Thesis Heleen Brüggen},
    pdfkeywords={},
    bookmarksnumbered=true,
    bookmarksopen=true,
    bookmarksopenlevel=1,
    colorlinks=false,
    pdfstartview=Fit,
    pdfpagemode=UseNone
}

\singlespacing%

\begin{document}
\begin{titlingpage}
\begin{center}
\Huge\textbf{Master Thesis:  \\ Multilevel Multivariate Imputation by Chained Equations through Bayesian Additive Regression Trees} \\
\Large\textit{Methodology and Statistics for the Behavioural, Biomedical and Social Sciences}

\vspace{.5cm}

\normalsize\textit{Heleen Brüggen}

\vspace{11.5cm}

\begin{minipage}{.5\textwidth}
\begin{center}
        \includegraphics[width=10cm]{graphs/UU_logo_2021_EN_RGB.png}
\end{center}
\end{minipage}%

\vspace{.25cm}

\begin{minipage}{0.5\textwidth}
\begin{flushleft}

\textbf{Word count:} \\
\textbf{Candidate Journal:} \\
\textbf{FETC Case Number:} \\
\textbf{Supervisors:} \\
T. Volker MSc. \\
Dr. G. Vink \\
H. Oberman MSc.
\end{flushleft}
\end{minipage}%
\begin{minipage}{0.5\textwidth}
\begin{flushright}

4036 \\ %detex thesis/Thesis_Heleen_Brueggen.tex | wc -w
Computational Statistics \& Data Analysis \\
23-1778 \\
------------------------\\
Utrecht University \\
Utrecht University \\
Utrecht University
\end{flushright}
\end{minipage}

\end{center}
\end{titlingpage}

\newpage
\tableofcontents
\newpage
\section{Introduction}
Incomplete data is a common challenge in many fields of research. Frequently used ad hoc strategies to deal with missing data, such as complete case analysis or mean imputation, often lead to erroneous inferences in realistic situations. Missingness can follow a multivariate mechanism that may depend on observed data or even unobserved data, leading to biased estimates and inaccurate variance estimates when using one of these ad hoc strategies \citep{buurenFlexibleImputationMissing2018, kang2013, enders2017, austin2021, little2002}. Multiple imputation (MI;~\Citealp{rubin1987}) is proven to be an effective method for dealing with multivariate incomplete data supported by a considerable amount of methodological research \citep{mistlerComparisonJointModel2017, buurenFlexibleImputationMissing2018, enders2017, burgette2010, austin2021, audigier2018, vanbuuren2007, grund2021, hughes2014, little2002}.
% Rubin defined three of such missing data mechanisms: Missing Completely At Random (MCAR) where the cause of the missing data is unrelated to the data, Missing At Random (MAR) where the missing data is related to the observed data, and Missing Not At Random (MNAR) where the missing data may also be related to unobserved data \citep{rubin1976}.

MI separates the missing data problem from the analysis problem \citep{mistlerComparisonJointModel2017, buurenFlexibleImputationMissing2018, enders2017, burgette2010, austin2021, audigier2018, vanbuuren2007, grund2021, hughes2014, little2002, carpenter2013, bartlett2015}. A statistical model specifying the variables used for imputation, i.e. the imputation model, is defined for every variable with missing values. Each missing value in the dataset is imputed \textit{m} times by drawing values from their posterior predictive distribution conditional on the observed data and parameters from the imputation model. By repeatedly drawing values from the posterior predicitve distributions -- in other words, the distribution of plausible replacement values -- the necessary variation associated with the missingness problem is considered. After imputation, each of the imputed datasets are analyzed according to the model of interest, i.e. the substantive analysis model. Then, their \textit{m} corresponding model parameters are pooled together according to Rubin's rules \citep{rubin1987}. 
% \citep{buurenFlexibleImputationMissing2018, austin2021, carpenter2013,enders2017}. 
One central requirement for MI is the concept of congeniality; the imputation model should should be at least as general as the analysis model and preferably all-encompassing \citep{grund2018, enders2018, meng1994multiple, bartlett2015, grund2016, little2002}. If not, the imputation model will not be compatible with the analysis model and the pooled estimates of the latter may be biased. 

% So, when the complexity of data increases, specifying the imputation models becomes more difficult \citep{grund2018, buurenFlexibleImputationMissing2018}.
When MI is applied in a multilevel data context, concerns regarding the concept of congeniality become more pronounced \citep{mistlerComparisonJointModel2017, enders2018, enders2018a, enders2020, buurenFlexibleImputationMissing2018, taljaard2008, enders2016, resche-rigon2018, audigier2018, dong2023, grund2016, grund2018a, grund2018, ludtke2017, grund2021, quartagno2022}. Multilevel data is hierarchically structured, where, for example, students are nested within classes within schools or patients within hospitals \citep{hox2017, hox2011}. When analyzing multilevel data, this hierarchical structure should be taken into consideration. Ignoring it will underestimate the intra-class correlation (ICC) and standard errors, as conventional statistical analyses assume independence of observations \citep{buurenFlexibleImputationMissing2018, ludtke2017, taljaard2008, hox2011}. The ICC can be interpreted as the proportion of the total variance at level-2 \citep{gulliford2005, shieh2012, hox2011}. Accounting for this structure, can be done using multilevel models (MLMs; \Citealp{hox2017, hox2011, ludtke2017}). MLMs can contain variables relating to the individual level -- level-1 variables -- or to the grouping structure -- level-2 variables or potentially higher order structures. For example, imagine a case where students are nested within classes. Here, the academic performance of a student is a level-1 variable, whereas the teacher's experience is a level-2 variable. Additionally, MLMs allow you to specify random intercepts, indicating that some classes have students that significantly perform better or worse academically on average; random slopes, indicating that the relationship between the performance of students and the outcome variable differs between classes; and cross-level interactions, indicating that the effect of performance of students can differ with the teacher's experience \citep{hox2017, hox2011}. Typically, the complexity of the multilevel analysis model is built step-wise with non-linearities, meaning the analysis model is not determined beforehand: predictors, random intercepts, random slopes, and cross-level interactions are added in a stepwise manner to the model \citep{hox2017, hox2011}. Thus, ensuring congeniality for the imputation model can be complex, since the final analysis model is not pre-determined. Furthermore, including the hierarchical structure along with cross-level interactions or other complicated non-linearities in imputation models is quite challenging \citep{buurenFlexibleImputationMissing2018, burgette2010, hox2011}, also because very complex models might not converge \citep{buurenFlexibleImputationMissing2018}.

A popular and flexible implementation of MI in a multilevel context, is fully conditional specification (FCS), otherwise known as chained equations \citep{audigier2018, burgette2010, vanbuuren2007, grund2018a}. FCS employs univariate linear mixed models to account for the hierarchical structure of multilevel models \citep{mistlerComparisonJointModel2017, enders2018, resche-rigon2018} and iteratively imputes each incomplete variable conditional on observed and previously imputed variables \citep{mistlerComparisonJointModel2017, buurenFlexibleImputationMissing2018, enders2016, enders2018, enders2018a, hughes2014, grund2018a}. Furthermore, it can impute non-linearities, such as cross-level interactions, by using `passive imputation` or defining a separate imputation model for the non-linearities \citep{buurenFlexibleImputationMissing2018, grund2018}. However, including these non-linearities in FCS is still very complicated \citep{grund2021, grund2018,buurenFlexibleImputationMissing2018}. FCS can also handle random intercepts and slopes, yet, once again, correctly specifying an imputation model accounting for these random effects can be challenging \citep{grund2021, grund2018,buurenFlexibleImputationMissing2018}.

%  and, thus, researchers' focus has predominantly been on the inclusion of random intercepts and slopes, but not of cross-level interactions \citep{grund2018a, grund2016, enders2018, enders2018a, enders2020, enders2016}.

Non-parametric, tree-based models might alleviate these complexities when defining imputation models. They do not assume a specific data distribution. So, they implicitly model non-linear relationships and can simultanously handle continuous and categorical variables \citep{hill2020, burgette2010, lin2019, chipman2010, james2021, salditt2023, breiman1984}. Studies showed that the use of tree-based, non-parametric models like regression trees, random forests, or Bayesian Additive Regression Trees (BART) in imputation of single-level data simplified the imputation process \citep{burgette2010,xu2016,silva2022,waljee2013}. They showed better model parameter estimates than parametric methods. Specifically, the imputations showed better confidence interval coverage of the parameters, lower variance and lower bias, especially in non-linear and interactive contexts \citep{burgette2010, xu2016, silva2022}. \citet{waljee2013} also found lower missclassification error rate for the predicted class as well as lower imputation error when imputing with a random forest algorithm compared to multivariate imputation by chained equations (\texttt{mice}) using linear, logistic, and polytomous logistic regression imputation models, K-nearest neighbors (KNN) and mean imputation.

In prediction, multilevel-BART models (M-BART) have predominantly been implemented with random intercepts only \citep{chen2020, wagner2020, tan2016, wundervald2022}. \citet{wagner2020} have found that this random intercept M-BART model provided better predictions with a lower mean squared error (MSE) compared to a parametric MLM, \citet{tan2016} found higher area under the curve (AUC) values compared to a singel-level BART model and linear logistic random intercept model, and \citet{chen2020} found better predictions and better coverage of the parameter estimates compared to parametric models and a single-level BART model. Other researchers modeled the random intercept as an extra split on each terminal node and found a lower MSE compared to a standard BART model and parametric MLMs \citep{wundervald2022}.~\citet{dorie2022} developed a multilevel BART model that included random intercepts, random slopes and cross-level interactions by modeling these random parts with a Stan \citep{lee2017} model and the fixed parts with a BART model. Their results showed that their algorithm \texttt{stan4bart} showed better coverage of the population values and lower root mean squared error (RMSE) compared to BART models with varying intercept, BART models ignoring the multilevel structure, bayesian causal forests, and parametric MLMs. 

% BART models have been implemented in a multilevel prediction context. However, multilevel-BART models (M-BART) have predominantly been implemented with only random intercepts \citep{chen2020, wagner2020, tan2016, wundervald2022}. In a prediction context, 

% \citet{wagner2020} have found that this random intercept M-BART model provided better predictions with a lower mean squared error (MSE) compared to a parametric MLM, \citet{tan2016} found higher area under the curve (AUC) values compared to a singel-level BART model and linear random intercept model, and \citet{chen2020} found better predictions and better coverage of the estimates compared to parametric models and a single-level BART model. Other researchers modeled the random intercept as an extra split on each terminal node and found a lower MSE compared to a standard BART model and parametric MLMs \citep{wundervald2022}. 

Despite these promising findings, M-BART models have yet to be implemented in a multilevel multiple imputation context. Thus, my research question will be: \textit{Can multivariate imputation by chained equations through a multilevel bayesian additive regression trees model improve the bias, variance, and coverage of the mulitlevel model parameter estimates compared to current practices?} Given the success of non-parametric models in single-level MI, I anticipate that employing M-BART models in a multilevel missing data context will reduce bias, accurately model variance, and improve estimate coverage compared to conventional implementations of multilevel MI, single-level MI, and complete case analysis in the R-package \texttt{mice} \citep{buuren2011}. 

\section{Method}
\subsection{Theoretical background}
\subsubsection{Bayesian Additive Regression Trees (BART)} \label{sec:bart}
BART is a sum-of-trees model proposed by \citet{chipman2010} with regression trees as its building blocks \citep{chipman2010, hill2020, james2021}. Regression trees recursively split the data into binary subgroups based on the predictors included in the model. At each step down the tree, these splits are based on the predictor that minimizes the variability within the subgroups from all predictors. Observations are then assigned to a certain subgroup according to these splits. This is continued until a certain stopping criterion is reached; for example, we desire a minimal number of observations with in a subgroup \citep{hastie2017, james2021, salditt2023, breiman1984}. Recursive binary partitioning of the predictor space doesn't assume a specific data form. This making regression trees, and as a consequence, BART, non-parametric models \citep{hastie2017, james2021, salditt2023, breiman1984} and allows regression trees to model non-linearities and other complicated relationships well and automatically \citep{hill2020, burgette2010}.
\citet{chipman2010} define the BART model as:
\begin{align}
\label{eq:BART}
f(\textbf{x}) &= \sum^{K}_{k=1}g(\textbf{x}; T_{k}, M_{k}),
\end{align} where $f(\mathbf{x})$ is the overall fit of the model: the sum of $K$ regression trees, $\textbf{x}$ are the predictor variables, $T_{k}$ is the k\textsuperscript{th} tree and $M_{k}$ is the collection of leaf parameters within the k\textsuperscript{th} tree, i.e. the collection of predictions for its terminal nodes \citep{chipman1998, chipman2006, chipman2010, hill2020, james2021}. The data are assumed to arise from a model with additive normally distributed errors: $Y = \sum^{K}_{k=1}g(\textbf{x}; T_{k}, M_{k}) + \epsilon, \epsilon \sim \mathcal{N}(0,\,\sigma^{2})$.
Next to the sum-of-trees model, BART also includes a regularization prior that constrains the size and fit of each tree so that each contributes only a small part of the variation in the outcome variables to prevent overfitting. The prior is imposed over all parameters of the sum-of-trees model, specifically, $(T_1, M_1), \dots, (T_K, M_K)$ and $\sigma$. However, the specification of the regularization prior is simplified by a series of independence assumptions: 
\begin{align}
\begin{split}
    \label{eq:independence_prior}
    p((T_1, M_1), \dots, (T_K, M_K), \sigma) &= \Big[\prod_{k}p(T_k, M_k)\Big]p(\sigma), \\
    &= \Big[\prod_{k}p(M_k|T_k)p(T_k)\Big]p(\sigma), \\
    p(M_k|T_k) &= \prod_{j}p(\mu_{jk}|T_k),
\end{split}
\end{align} where $\mu_{jk} \in M_k$. These assumptions state that the trees ($T_{k}$), leaf parameters ($\mu_{j}|T_{k}$), and the standard deviation ($\sigma$) are independent of each other. Thus, priors only need to be specified for those parameters \citep{chipman2010, hill2020, chipman2006, chipman1998}.~\citet{chipman1998} define an independent prior for each tree. The probability that a node at depth $d$ splits is defined as: 
\begin{align}
\label{eq:tree_prior}
    \alpha(1+d)^{-\beta}, \alpha \in (0,1), \beta \in [0, \infty),
\end{align} where the default specification put forth by~\citet{chipman2006,chipman2010} is $\alpha = .95$ and $\beta = 2$. This specification sets the probability of a tree with 1, 2, 3, 4, and 5 nodes at .05, .55, .28, .09, and .03 respectively. Thus, smaller trees are favoured.~\citet{chipman2006,chipman2010} also provide a default specification for the prior for the leaf parameters. They propose to rescale the response value to the interval $[-.5,.5]$. Then, the leaf parameter prior is defined as: 
\begin{align}
\label{eq:leaf_prior}
    \mu_{jk} \sim \mathcal{N}(0, \sigma^2_{\mu}), \text{with } \sigma^2_{\mu} = \frac{.5}{t\sqrt{K}},
\end{align} where $t$ is a preselected number and $K$ is the number of trees. This prior shrinks the tree parameters $\mu_{jk}$ towards 0, decreasing the effect of the individual tree components. If $t$ or $K$ increase, more shrinkage is applied.~\citet{chipman2006,chipman2010} found good results with and recommend using $t = 2$ -- or values between 1 and 3 -- as a default choice. Furthermore, \citet{chipman2006,chipman2010} propose the conjugate inverse chi-square distribution as the prior for the residual standard deviation $\sigma^2 \sim \nu\lambda/\chi^{2}_{\nu}$. They represent the degrees of freedom, $\lambda$, as the probability that the BART residual standard deviation, $\sigma$, is less than the estimated residual standard deviation from a linear regression model, $\hat{\sigma}_\text{OLS}$. Their default specification of the hyperparameters is $\nu = 3$ and $\text{Pr}(\sigma < \hat{\sigma}_\text{OLS}) = .9$ \citep{chipman2010, hill2020, chipman2006, chipman1998}.

BARTs are estimated using the Bayesian back-fitting Markov Chain Monte Carlo (MCMC) algorithm \citep{chipman2010, hill2020, chipman2006, chipman1998,james2021}. Each tree is intialized with a single root node with the mean response value divided by the number of trees ($\hat{f}^1_k(x) = \frac{1}{nK}\sum_{i = 1}^{n}y_i$, with sample size $n$). Then, each pair $(T_k, M_k)$ is updated considering the remaining trees, their associated parameters, and the residual standard deviation ($\sigma$) by sampling from the following conditional distribution: 
\begin{align}
\label{eq:backfitting}
    (T_k, M_k)|T_{k'}, M_{k'}, \sigma, y.
\end{align} However, this conditional distribution only depends on ($T_{k'}, M_{k'}, y$) through the partial residuals:
\begin{align}
    \label{eq:partialresiduals}
    r_i &= y_i - \sum_{k' < k} \hat{f}^{b}_{k'}(x_{i}) - \sum_{k' > k} \hat{f}^{b-1}_{k'}(x_{i}), \text{with } i = 1, \dots, n,
    \end{align} where $\hat{f}^{b}_{k}(x_{i})$ is the prediction of the $k$\textsuperscript{th} tree in the $b$\textsuperscript{th} iteration for person $i$ and sample size $n$. Thus, updating each pair $(T_k, M_k)$ simplifies to proposing a new tree fit to the partial residuals, $r_{i}$, treating them as the data, by perturbing the tree from the previous iteration. Perturbations entail either \textit{growing}, \textit{pruning}, or \textit{changing} a tree.~\textit{Growing} means adding additional splits, \textit{pruning} removes splits, and \textit{changing} changes decision rules. The algorithm stops after the specified number of iterations \citep{chipman2010, hill2020, chipman2006, chipman1998, james2021}.

\subsubsection{Multilevel-BART (M-BART)}
\citet{wagner2020,tan2016} and \citet{dorie2024} define an M-BART model including a random intercept. The BART model (\ref{eq:BART}) is extended to include a random intercept by: 
\begin{align} 
    \label{eq:M-BART}
    m(\textbf{x}) &= \sum^{K}_{k=1}g(\textbf{x}; T_{k}, M_{k}) + \alpha_{j}, 
\end{align} where, now, $f(\textbf{x})$ is the overall fit of the model incorporating random intercept $\alpha_{j}$ for cluster $j$ and. So, the data are now assumed to arise from the following model: 
\begin{align}
    \label{eq:M-BART_model}
    Y_{ij} = \sum^{K}_{k=1}g(\textbf{x}; T_{k}, M_{k}) + \alpha_j + \epsilon_{ij}, & &\epsilon_{ij} \sim \mathcal{N}(0,\,\sigma^{2}), \hspace{.2cm} \alpha_{j} \sim \mathcal{N}(0,\,\tau^{2}),
\end{align} where $\alpha_j \perp \epsilon_{ij}$. Now the joint prior distribution (\ref{eq:independence_prior}) becomes: 
\begin{align}
\begin{split}
    \label{eq:indepdence_prior_mbart}
    p((T_1, M_1), \dots, (T_K, M_K), \sigma) &= \Big[\prod_{k}p(T_k, M_k)\Big]p(\sigma)p(\tau), \\
    &= \Big[\prod_{k}p(M_k|T_k)p(T_k)\Big]p(\sigma)p(\tau), \\
    p(M_k|T_k) &= \prod_{j}p(\mu_{jk}|T_k).
\end{split}
\end{align} 
A Metropolis within Gibbs procedure is used to draw values from the posterior. First, the Gibbs sample for $\sigma$, $\tau$, and $\alpha_j$ are obtained from their respective posterior distributions. Then, we obtain $\tilde{Y}_{ij} = Y_{ij} - \alpha_{j}$ and view $\tilde{Y}_{ij}| \boldsymbol{X}_{j}$ as a BART model. So, $\tilde{Y}$ is now used as the outcome variable in the BART algorithm described in the previous section,~\ref{sec:bart}.~\citep{wagner2020,tan2016}.~\citet{dorie2024} implemented this algorithim within the R-package \texttt{dbarts} with the funtion \texttt{rbart\_vi()}. Where, the default prior for the random intecept is $\tau \sim \text{Cauchy}(0, 2.5)$: a Cauchy distribution with a scale parameter 2.5 times the original scale.

\subsubsection{stan4bart}
\citet{dorie2022} developed a multilevel BART model that included random intercepts, random slopes, and cross-level interactions. They extend a Bayesian linear, mixed model with a BART model (\ref{eq:BART}). The resulting model is:
\begin{align}
    \label{eq:stan4bart}
    h(\textbf{x}) &= \mathbf{x}^{\beta}\boldsymbol{\beta} + f(\mathbf{x}; T_{K}, M_{K}) + \boldsymbol{\lambda}\mathbf{w},
\end{align} where $\mathbf{x}^{\beta}$ is a vector of 1 -- for the intercept -- and the linear predictors; $\boldsymbol{\beta}$ is a vector of linear, parametric coefficients; $\boldsymbol{\lambda}$ is a vector of all parametric random slopes and intercepts; $\mathbf{w}$ is a vector of the coefficients for the random slopes and intercepts; and $f(\mathbf{x}; T_{K}, M_{K})$ is a non-parametric, sum-of-trees BART model \citep{dorie2022}. So, the data are assumed to arise from the following model:
\begin{align}
    \label{eq:stan4bart_model}
    Y_{ij} = \mathbf{x}^{\beta}\boldsymbol{\beta} + f(\mathbf{x}; T_{K}, M_{K}) + \boldsymbol{\lambda}\mathbf{w} + \epsilon_{ij}, & &\epsilon_{ij} \sim \mathcal{N}(0,\,\sigma^{2}), \hspace{.2cm} \boldsymbol{\lambda} \sim \mathcal{N}(0,\,\boldsymbol{\Sigma}_{\lambda}),
\end{align} where $\boldsymbol{\Sigma}_{\lambda}$ is the variance-covariance matrix for the random intercept and slopes. The model is implemented as a Gibbs sampler: a  Hamiltonian Monte Carlo, no-U-turn sampler with a diagonal Euclidean adaptation matrix is used to jointly sample the linear, parametric components given the non-parametric components. The non-parametric components are sampled using the BART algorithm described in section~\ref{sec:bart} \citep{dorie2022}. To accomplish this, a parametric Stan model \citep{lee2017} fits equation \ref{eq:stan4bart} with $f(\mathbf{x}; T_{K}, M_{K})$ as a generic linear offset. \citet{dorie2022} combine a custom mutable Stan sampler object with a BART sampler with a fixed variance and offset term: First, the Stan sampler collects the current draws of the BART model into $\text{vec}_if(\mathbf{x}_{i}; T_{K}, M_{K})$ and uses this to draw $\mathbf{\beta}, \mathbf{\lambda}, \sigma, \Sigma_\lambda | \mathbf{Y}, \text{vec}_if(\mathbf{x}_{i}; T_{K}, M_{K})$. Then, $\sigma$ and $\text{vec}_i\big[\mathbf{x}_i^{\beta}\boldsymbol{\beta} + \boldsymbol{\lambda}\mathbf{w}_i\big]$ are passed to BART, which produces $M_k,T_k | \mathbf{Y}, \text{vec}_i\big[\mathbf{x}_i^{\beta}\boldsymbol{\beta} + \boldsymbol{\lambda}\mathbf{w}_i\big], \sigma, M_{k'}, T_{k'}$. Then, the cycle is completed by passing $\text{vec}_if(\mathbf{x}_{i}; T_{K}, M_{K})$ back to Stan. The process is continued for the set amount of posterior samples intended for inference. This algorithm is implemented in the R-package \texttt{stan4bart} \citep{dorie2023a}.

\subsection{Simulation study}
\subsubsection{Data generating mechanism}
We assembled a simulation study to evaluate the performance of multilevel BART models in a multilevel imputation context. The population data-generating mechanism is based on the following MLM:
\begin{subequations}
\label{eq:population}
\begin{align}
        y_{ij} &= \beta_{0j} + \sum_{k=1}^{7}\beta_{kj}X_{kij} + \epsilon_{ij}, \hspace{3cm} X_{kij} \sim \mathcal{MVN}(0, \boldsymbol{\Sigma}_{x}), \label{eq:population1} \\
        &\hspace{.5cm}\beta_{0j} = \gamma_{00} + \sum_{p=1}^{2}\gamma_{0q}Z_{pj} + \upsilon_{0j}, \label{eq:beta0}\\
        &\hspace{.5cm}\beta_{kj} = \gamma_{k0} + \sum_{p=1}^{2}\gamma_{kq}Z_{pj} + \upsilon_{kj}, \hspace{2cm} Z_{pj} \sim \mathcal{MVN}(0, \boldsymbol{\Sigma}_{z}), \label{eq:betak}
\end{align}
\end{subequations} where $y_{ij}$ is a continuous level-1 outcome variable for person $i$ in group $j$ and $X_{kij}$ are 7 continuous level-1 variables and $Z_{pj}$ are 2 continuous level-2 variables. The predictors are multivariate normally distributed with means of 0 and variance-covariance matrix $\boldsymbol{\Sigma}_{x}$ and $\boldsymbol{\Sigma}_{z}$, respectively:
\begin{subequations}
\begin{align}
    \boldsymbol{\Sigma}_{x} &= \begin{pmatrix}
        6.25& & & & & & \\
        2.25& 9& & & & & \\
        1.5& 1.8& 4& & & & \\
        2.25& 3.06& 2.04& 11.56& & & \\
        1.5& 1.8& 1.2& 2.04& 4& & \\
        1.125& 1.35& 0.9& 1.53& .9& 2.25& \\
        3.3& 3.96& 2.64& 4.488& 2.64& 1.98& 19.36
    \end{pmatrix}, \label{eq:sigma.x} \\
    \boldsymbol{\Sigma}_{z} &= \begin{pmatrix}
        1& \\
        .48& 2.56
    \end{pmatrix}. \label{eq:sigma.z}
\end{align}
\end{subequations}
The covariances between the variables are calculated such that the correlation between the variables is .3, aligned with Cohen's \citeyearpar{cohen1990} medium effect size benchmark. The residuals are normally distributed as,
\begin{align}
    \epsilon_{ij} \sim \mathcal{N}(0, 25).
\end{align}
The random intercept $\beta_{0j}$ is determined by the overall intercept $\gamma_{00}$, the 2 group-level effects $\gamma_{0q}Z_{pj}$ and the group-level random residuals $\upsilon_{0j}$. The overall intercept $\gamma_{00}$ is set to 10 and the group-level effects $\gamma_{01}$ and $\gamma_{02}$ to .5.
The 7 regression coefficients $\beta_{kj}$ for the continuous variables $X_{kij}$ depend on the intercepts $\gamma_{k0}$, the cross-level interactions $\gamma_{kp}Z_{pj}$, and the random slopes $\upsilon_{kj}$. The 7 intercepts, or within-group effect sizes, $\gamma_{k0}$ are set to .5, the cross-level interactions $\gamma_{11}$, $\gamma_{21}$, and $\gamma_{32}$ are set to .35.
\begin{align}
    \gamma_{00} = 10, \hspace{.2cm} \boldsymbol{\gamma}_{0p} = \begin{pmatrix}
        .5 \\ .5
        \end{pmatrix}, \hspace{.2cm} \boldsymbol{\gamma}_{k0} = \begin{pmatrix}
        .5 \\ .5 \\ .5 \\ .5 \\ .5 \\ .5 \\ .5
        \end{pmatrix}, \hspace{.2cm} \boldsymbol{\gamma}_{kp} = \begin{pmatrix}
        .35 & 0 \\ .35 & 0 \\ 0 & .35 \\ 0 & 0 \\ 0 & 0 \\ 0 & 0 \\ 0 & 0
        \end{pmatrix}.
\end{align}
The random slopes are multivariate normally distributed with a mean of 0 and a variance-covariance matrix $\mathbf{T}$ shown in equation~\ref{eq:T}. Again, the covariances are calculated to yield a correlation of .3.
\begin{align}
    \boldsymbol{\upsilon}_{j} \sim \mathcal{MVN}(0, \mathbf{T}), \hspace{.2cm}
    \mathbf{T} = \begin{pmatrix}
        t_{00}& & & & & & \\
          .3& 1& & & & & \\
          .3& .3& 1& & & & \\
          .3& .3& .3& 1& & & \\
          0& 0& 0& 0& 0& & \\
          0& 0& 0& 0& 0& 0& \\
          0& 0& 0& 0& 0& 0& 0 \\
          0& 0& 0& 0& 0& 0& 0& 0
    \end{pmatrix} \label{eq:T}.
\end{align}
The variance of $\upsilon_{0j}$, the group-level random residuals $t_{00}$, are scaled such that the specified ICC values as in table~\ref{tab:simulationparameters} was obtained. The following formula is used to calculate $\upsilon_{0j}$ following the variance decomposition from~\cite{rights2019}:

\begin{align}
\label{eq:variancedecomposition}
\text{ICC} = \frac{\boldsymbol{\gamma}^{b'}\boldsymbol{\phi}^{b}\boldsymbol{\gamma}^{b} + \tau_{00}}{\boldsymbol{\gamma}^{w'}\boldsymbol{\phi}^{w}\boldsymbol{\gamma}^{w} + \boldsymbol{\gamma}^{b'}\boldsymbol{\phi}^{b}\boldsymbol{\gamma}^{b} + tr(\mathbf{T}\boldsymbol{\Sigma})+ \tau_{00} + \sigma^{2}},
\end{align} where $\boldsymbol{\gamma}^{b}$ and $\boldsymbol{\gamma}^{w}$ are the level-1 and level-2 fixed effects; $\boldsymbol{\phi}^{b}$ is the variance-covariance matrix of a vector with 1, for the intercept, and all level-2 predictors; $\boldsymbol{\phi}^{w}$ is the variance-covariance matrices of all cluster-mean-centered level-1 predictors; $\tau_{00}$ is the variance of the random intercept; $\mathbf{T}$ is the variance-covariance matrix of the random intercept and slopes; $\boldsymbol{\Sigma}$ is the variance-covariance matrix of a vector containing 1, for the intercept, and the level-1 variables; and $\sigma^{2}$ is the residual variance. The value for $\tau_{00}$ is calculated using the function \texttt{uniroot()} in R \citep{rcoreteam2023}.

\subsubsection{Simulation design} 
Table~\ref{tab:simulationparameters} shows the design factors considered in the simulation study. These factors are either grounded in prior research or deemed realistic in real-world applications \citep{gulliford1999, murray2003, hox2017, grund2018, enders2018a, enders2020}. According to \citet{kreft2007}, 30 groups is the smallest acceptable number in multilevel research and 50 groups is frequent in organizational research \citep{maas2005}. Group sizes of 15 are typical in edicational research \citep{ludtke2017} and group sizes of 50 are often used in simulation studies \citep{maas2005,enders2018,akkayahocagil2023,grund2018,enders2018a,enders2020}. The ICC was chosen to be .5, which is often used as an upper limit in methodological research \citep{enders2020,enders2018,enders2018a,mistler2017,grund2018,salditt2023}.~\citet{oberman2023} recommend including both Missing Completely At Random (MCAR) and Missing At Random (MAR) missingness mechanisms in simulation studies. They pose that the statistical properties of the imputation method are not deemed sound if it cannot yield valid inferences under MCAR. Furthermore, they pose that including observed-data-dependend missingness -- for example, MAR -- is of utmost importance in evaluating the imputation method's performance. The amount of missingness in data sets is varied between 0\% and 50\%. 0\% missingness is included as an additional benchmark and 50\% missingess is often used in simulation studies as a high amount of missingness \citep{ludtke2017,grund2016,schouten2021}. For each combination of design factors, 100 datasets are simulated. 5 different imputation methods are compared: 
\begin{enumerate}
    \item conventional single-level imputation with PMM (predictive mean matching),
    \item conventional multilevel imputation with PMM,
    \item single-level BART imputation,
    \item multilevel BART imputation accounting for random intercepts \citep{chen2020, wagner2020, tan2016},
    \item multilevel BART imputation accounting for random effects and cross-level interactions \citep{dorie2022}.
\end{enumerate} 
\begin{wraptable}{r}{8cm}
\centering
\caption{Simulation design}
\label{tab:simulationparameters}
\begin{tabular}{l|c}
        \textbf{Parameter}                                  & \textbf{Values} \\ \hline
        Number of clusters (J))                             & 30, 50 \\
        Within-cluster sample size (n\textsubscript{j})     & 15, 50 \\
        Intraclass Correlation (ICC)                        & .5 \\
        Missing data mechanism                              & MCAR, MAR \\
        Amount of missingness                               & 0\%, 50\%
\end{tabular}
\end{wraptable} The first and second methods are implemented with the R-packages \texttt{mice} and \texttt{miceadds} \citep{robitzsch2024}. The conventional single-level imputation is implemented with the imputation method \texttt{pmm} and the conventional multilevel imputation is implemented with the \texttt{2l.pmm} method for level-1 variables and \texttt{2lonly.mean} for level-2 variables.

The third, single-level BART, fourth, random intercept BART and fifth method, multilevel BART methods are implemented by writing new method-functions in R \citep{rcoreteam2023} for the package \texttt{mice}. The functions \texttt{bart} and \texttt{rbart\_vi} from the \texttt{dbarts} package were used for the single-level and random intercept BART imputation methods \citep{dorie2024}. The function \texttt{stan4bart} from the package \texttt{stan4bart} was used for the multilevel BART imputation method accounting for random effects and cross-level interactions \citep{dorie2023a}. The functions were written such that they can be used as imputation methods in the \texttt{mice} package. All three functions are implementend as follows: for every variable to be imputed, a respective BART model is fitted based on the predictor matrix. Then, the fitted values -- the posterior means -- are extracted for the observed and missing values. Imputations for the missing values are then obtain using predictive mean matching by matching the predicted values for the observed cases to the predicted values for the missing cases. The code for these functions can be found in the appendix -- listing~\ref{lst:singlelevelBART},~\ref{lst:randominterceptBART}, and~\ref{lst:multilevelBART}.

For all imputation methods, the incomplete data sets are imputed 5 times with 10 iterations each. Then, each of the 5 imputed datasets are then analyzed using the R-package \texttt{lme4} \citep{bates2015} with an MLM reflecting the population generating mechanism: \texttt{y = 1 + x1 + x2 + x3 + x4 + x5 + x6 + x7 + z1 + z2 + x1 * z1 + x2 * z1 + x3 * z2 + (1 + x1 + x2 + x3 | group)}. The estimates from the 5 imputed datasets are pooled together using the R-package \texttt{mice} \citep{buuren2011}. These pooled estimates are compared on the bias, coverage, and the width of the 95\% confidence intervals.

As an additional benchmark, the imputation methods will also be compared to analyses using listwise deletion, i.e. complete case analysis, and using the true data without missing values.

\subsubsection{Missing data generation}
Missing values in the variables are introduced by multivariate amputation using the function \texttt{ampute()} \citep{schouten2018} from package \texttt{mice}. As can be seen in table~\ref{tab:simulationparameters}, the missing data mechanism is either Missing Completely At Random (MCAR) or Missing At Random (MAR). The missing data mechanism is said to be MCAR when the cause of the missing data is unrelated to the data and MAR when the missing data is related to the observed data \citep{rubin1976}. The amount of missingness is either 0\% or 50\%, which is defined as the percentage of cases that have at least one missing value. 

For both MCAR and MAR, all possible patterns with 1 to 5 missing values out of the 10 variables (\textit{x1, x2, x3, x4, x5, x6, x7, z1, z2, and y}) per case are generated. They have the same relative frequency of occurence in the data sets. So, 50\% of the cases had 1 to 5 missing values. 

For the MAR mechanism, the weighted sum of scores on the observed variables is used to predict the probability of missingness for a case. The weights of the variables \textit{x4} and \textit{z1} are set to 2 and 1.5 respectively when they remain observed in a specific pattern, while the weights of the other variables that remain observed in a specific pattern are set to 1. The type of missingess is set to \texttt{`RIGHT'} meaning that cases with a higher weighted sum of scores have a higher probability of becoming incomplete. So, this means that cases with higher values on \textit{x4} and \textit{z1} are more likely to become incomplete.

In summary, either no missing values are introduced (0\%), or up to 5 missing values are introduced in 50\% of the cases. When data is MAR, the probability of a value being missing depends on the observed values of all other variables, with variables \textit{x4} and \textit{z1} having a greater influence on this probability.

\subsubsection{Evaluation}
The estimates from the analysis models are evaluated in terms of absolute bias, coverage of 95\% confidence intervals, with their respective Monte Carlo SE (MCSE), and the width of the 95\% confidence intervals \citep{morris2019,oberman2023}:
\begin{align}
    \text{Bias} &= \frac{1}{n_{\text{sim}}} \sum_{t=1}^{n_{\text{sim}}} (\hat{\theta}_t - \theta), &
    \text{MCSE}_{\text{Bias}} &= \sqrt{\frac{\sum_{t=1}^{n_{\text{sim}}} (\hat{\theta}_t - \bar{\theta})^2}{n_{\text{sim}}(n_{\text{sim}}-1)}}, \label{eq:bias} \\
    % \text{MSE} &= \frac{1}{n_{\text{sim}}} \sum_{t=1}^{n_{\text{sim}}} (\hat{\theta}_t - \theta)^{2}, &
    % \text{MCSE}_{\text{MSE}} &= \sqrt{\frac{\sum_{t=1}^{n_{\text{sim}}} [(\hat{\theta}_t - \theta)^2 - \hat{\text{MSE}}]^2}{n_{\text{sim}}(n_{\text{sim}}-1)}}, \label{eq:mse}\\
    \text{Coverage} &= \frac{1}{n_{\text{sim}}} \sum_{t=1}^{n_{\text{sim}}} 1(\hat{\theta}_{\text{low,i}} \leq \theta \leq \hat{\theta}_{\text{upp,i}}), &
    \text{MCSE}_{\text{Coverage}} &= \sqrt{\frac{\hat{\text{Coverage}}(1-\hat{\text{Coverage}})}{n_{\text{sim}}}}, \label{eq:coverage} \\
    \text{CIW} &= \frac{1}{n_{\text{sim}}} \sum_{t=1}^{n_{\text{sim}}} (\hat{\theta}_{\text{upp,i}} - \hat{\theta}_{\text{low,i}}), \label{eq:width}
\end{align} where $\hat{\theta}_t$ is the estimated parameter in simulation \textit{t}, $\theta$ is the true value, $\bar{\theta}$ is the mean of $\hat{\theta}_t$, and $n_{\text{sim}}$ is the number of simulated datasets. The lower and upper bounds of the 95\% confidence intervals are denoted as $\hat{\theta}_{\text{low,i}}$ and $\hat{\theta}_{\text{upp,i}}$ respectively. The coverage is the proportion of the 95\% confidence intervals that contain the true value. 

\citet{morris2019,enders2018,oberman2023,buurenFlexibleImputationMissing2018} suggest that a coverage of 95\% is acceptable. Poor coverage, i.e. below 95\%, indicates biased estimates or too narrow intervals. While, coverage above 95\% indicates that efficiency could still be gained. The width of the confidence intervals is a measure of the statistical precision of the estimates: a smaller width indicates a more precise estimate \citep{oberman2023,buurenFlexibleImputationMissing2018}.

\section{Results}
\graphicspath{{./graphs/}}

\subsection{Bias}
Figure \ref{fig:bias1} shows the absolute bias of the estimates of the linear mixed model -- except the residual variance, $\epsilon_{ij}$, and the intercept variance, $\upsilon_{0}$ for interpretabiliy -- for all imputation methods in consideration. The absolute bias of the residual variance and the intercept variance are shown in figure \ref{fig:bias2}.

First, the estimates of the fixed effects -- the overal intercept, $\gamma_{00}$; level-1 effects $\gamma_{10}:\gamma_{70}$; level-2 effects $\gamma_{01}$ and $\gamma_{02}$; and the cross-level interactions $\gamma_{11}, \gamma_{21}$, and $\gamma_{32}$ -- will be considered in terms of absolute bias. 

Looking at when the data is MAR, figure \ref{fig:bias1} shows that when there are more groups in the data set, the intercept tends to be underestimated for all imputation methods. Also, looking at listwise deletion, the intercept is underestimated for all number of groups and group sizes compared to the true data analysis. The single-level imputation methods -- PMM and BART --, show little bias for 30 groups of sizes 15 and 50. However, when increasing the amount of groups to 50, PMM and BART increase in negative bias -- underestimation -- for the intercept. On top of that, only the MCSE of BART includes the zero-bias line with 50 groups of size 50. From the multilevel imputation methods, 2l.PMM and R-BART show a similar pattern: their bias is small when the number of groups is 30 and increases when the number of groups is 50. Stan4bart tends to overestimate the intercept when the groups sizes are 50, while underestimating it when they are 15. When the data is MCAR, the bias of the intercept for all methods is small when the number of groups is large. Furthermore, the imputation methods perform very similar when the number of groups are 30: slightly overestimating the intercept when the group sizes are 15 and slightly underestimating it when the group sizes are 50.

The level-1 effects show no clear pattern in bias for the different factor designs. Overall, all methods seem to perform well in terms of bias except R-BART, which seems to routinely underestimate some of the level-1 effects for all conditions. Furthermore, the simulation uncertainty -- i.e. the MCSE -- of the level-1 effects increases when the total sample sizes are smaller.

The level-2 effects 

\begin{figure}[H]
    \centering
    \includegraphics[width=1\textwidth]{bias1.png}
    \caption{Absolute bias of the estimates  of the linear mixed model with respective Monte Carlo SE for all simulated data sets over 100 simulations with ICC = .5 excluding the residual variance (eij) and intercept variance (u0).}
    \label{fig:bias1}
    % \caption*{\footnotesize{A note}}
\end{figure}

\begin{figure}[H]
    \centering
    \includegraphics[width=1\textwidth]{bias2.png}
    \caption{Bias of the eij and u0}
    \label{fig:bias2}
\end{figure}

\subsection{Coverage}
\begin{figure}[H]
    \centering
    \includegraphics[width=1\textwidth]{coverage.png}
    \caption{Coverage of the estimates}
    \label{fig:coverage}
\end{figure}

\subsection{Confidence interval width}
\begin{figure}[H]
    \centering
    \includegraphics[width=1\textwidth]{ciw.png}
    \caption{Confidence interval width of the estimates}
    \label{fig:ciw}
\end{figure}

\section{Discussion}

\section{Conclusion}

\newpage
\section{Appendix}
\begin{lstlisting}[language=R, caption = {Imputation function for single-level BART}, label = {lst:singlelevelBART}]
    mice.impute.bart <- function(y, ry, x, wy = NULL, use.matcher = FALSE, donors = 5L, ...) {
        install.on.demand("dbarts", ...)
        if (is.null(wy)) {
            wy <- !ry
        }
    
        # Parameter estimates
        fit <- dbarts::bart(x, y, keeptrees = TRUE, verbose = FALSE)
    
        yhatobs <- fitted(fit, type = "ev", sample = "train")[ry]
        yhatmis <- fitted(fit, type = "ev", sample = "train")[wy]
    
        # Find donors
        if (use.matcher) {
            idx <- matcher(yhatobs, yhatmis, k = donors)
        } else {
            idx <- matchindex(yhatobs, yhatmis, donors)
        }
    
        return(y[ry][idx])
    }
\end{lstlisting}
\begin{lstlisting}[language=R, caption={Imputation function for random intercept BART}, label={lst:randominterceptBART}]
    mice.impute.2l.rbart <- function(y, ry, x, wy = NULL, type, use.matcher = FALSE, donors = 5L, ...) {
        install.on.demand("dbarts", ...)
        if (is.null(wy)) {
            wy <- !ry
        }
    
        clust <- names(type[type == -2])
        effects <- names(type[type != -2])
        X <- x[, effects, drop = FALSE]
    
        model <- paste0(
            "y ~ ", paste0(colnames(X), collapse = " + ")
        )
    
        fit <- dbarts::rbart_vi(formula = formula(model), group.by = clust, data = data.frame(y, x), verbose = FALSE, n.threads = 1, n.samples = 500L, n.burn = 500L, ...)
    
        yhatobs <- fitted(fit, type = "ev", sample = "train")[ry]
        yhatmis <- fitted(fit, type = "ev", sample = "train")[wy]
    
        # Find donors
        if (use.matcher) {
            idx <- matcher(yhatobs, yhatmis, k = donors)
        } else {
            idx <- matchindex(yhatobs, yhatmis, donors)
        }
    
        return(y[ry][idx])
    }
\end{lstlisting}
\begin{lstlisting}[language=R, caption={Imputation function for multilevel BART with random effects and cross-level interactions}, label={lst:multilevelBART}]
    mice.impute.2l.bart <- function(y, ry, x, wy = NULL, type, intercept = TRUE, use.matcher = FALSE, donors = 5L, ...) {
        install.on.demand("stan4bart", ...)
        if (is.null(wy)) {
            wy <- !ry
        }
    
        if (intercept) {
            x <- cbind(1, as.matrix(x))
            type <- c(2, type)
            names(type)[1] <- colnames(x)[1] <- "(Intercept)"
        }
    
        clust <- names(type[type == -2])
        rande <- names(type[type == 2])
        fixe <- names(type[type > 0])
    
        lev <- unique(x[, clust])
    
        X <- x[, fixe, drop = FALSE]
        Z <- x[, rande, drop = FALSE]
        xobs <- x[ry, , drop = FALSE]
        yobs <- y[ry]
        Xobs <- X[ry, , drop = FALSE]
        Zobs <- Z[ry, , drop = FALSE]
    
        # create formula
        fr <- ifelse(length(rande) > 1,
            paste0("+ (1 +", paste(rande[-1L], collapse = "+")),
            " + (1 "
        )
        randmodel <- paste0(
            "y ~ bart(", paste0(fixe[-1L], collapse = " + "), ")",
            fr, "| ", clust, ")"
        )
    fit <- eval(parse(text = paste("stan4bart::stan4bart(", randmodel,
        ", data = data.frame(y, x),
            verbose = -1,
            bart_args = list(k = 2.0, n.samples = 500L, n.burn = 500L, n.thin = 1L, n.threads = 1))",
        collapse = ""
    )))
    
        yhatobs <- fitted(fit, type = "ev", sample = "train")[ry]
        yhatmis <- fitted(fit, type = "ev", sample = "train")[wy]
    
        # Find donors
        if (use.matcher) {
            idx <- matcher(yhatobs, yhatmis, k = donors)
        } else {
            idx <- matchindex(yhatobs, yhatmis, donors)
        }
    
        return(y[ry][idx])
    }
\end{lstlisting}

\newpage
\bibliography{thesis}
\addcontentsline{toc}{section}{References}
\bibliographystyle{apalike}
\end{document}

